\section{Cursos especializados}

\renewcommand{\listitemsymbol}{-~} % Changes the symbol used for lists

\cvitem{50 horas}{Python for Data Science and Machine Learning Bootcamp - UDEMY}
\cvitem{50 horas}{Procesamiento y análisis de datos geofísicos usando Python 3 - SENAMHI}
\cvitem{24 horas}{Modelo Hidrológico: Modelo de Grandes Cuencas MGB - SENAMHI - UFRGS Brasil}
\cvitem{40 horas}{Rescate de Datos Climáticos - SENAMHI}
\cvitem{20 horas}{Bayesian Statistics: From Concept to Data
	Analysis - Universidad de California, Santa Cruz - Coursera }
\cvitem{20 horas}{Building Data Visualization Tools with R - Universidad Johns Hopkins - Coursera }
\cvitem{20 horas}{Building R Packages - Universidad Johns Hopkins - Coursera }
\cvitem{20 horas}{Advanced R Programming - Universidad Johns Hopkins - Coursera}
\cvitem{20 horas}{The R Programming Environment - Universidad Johns Hopkins - Coursera}
\cvitem{10 horas}{Análisis del producto satelital de precipitación GPM para la región andina - SENAMHI}
\cvitem{10 horas}{Conducta responsable de investigación - QUIPU}
\cvitem{24 horas}{Introducción a SWAT - SENAMHI}
\cvitem{20 horas}{Python en hidrología - Gidahatari}
\cvitem{20 horas}{Programación en lenguaje R - PROMIDAT}
\cvitem{32 horas}{Geoestadística aplicado al mapeamiento de variables hidroclimática - SENAMHI}
\cvitem{32 horas}{Sensoramiento remoto aplicado a la hidrología de ríos amazónicos - SENAMHI}
\cvitem{32 horas}{Control de calidad, homogenización y metadata - SENAMHI}
\cvitem{40 horas}{Meteorología de montaña - SENAMHI}
\cvitem{40 horas}{Meteorología costera - SENAMHI}
\cvitem{30 horas}{Variabilidad hidroclimática y modelos estocásticos en Hidrología - UNALM}
\cvitem{32 horas}{Modelización regional del Clima y aplicaciones en el Perú - SENAMHI}
\cvitem{20 horas}{SIG en la gestión del agua - Gidahatari}
\cvitem{15 horas}{Interpretación de imágenes satelitales - CPTEC - INPE}

