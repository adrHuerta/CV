\section{Conferencias}

\cvitem{En. 2015}{\textbf{Huerta, A.}, and Lavado-Casimiro, W., (2015). Extremos de precipitación en la vertiente del Lago Titicaca, \textit{En Seminario de Estudios e Investigaciones Hidrológicas del Servicio Nacional de Hidrología del Perú - 2015}, Lima, Perú. \href{https://www.scribd.com/document/254504358/4-HUERTA-A}{https://www.scribd.com/document/254504358/4-HUERTA-A}.}

\cvitem{Oct. 2015}{\textbf{Huerta, A.}, (2015). Caracterización de extremos de precipitación en la vertiente del lago Titicaca, \textit{En 6th HYdro-geochemistry of the AMazonian Basin (HYBAM) Scientific Meeting}, Cuzco, Perú.}

\cvitem{Set. 2017}{\textbf{Huerta, A.}, (2017). Variabilidad espacio-temporal de las sequías meteorológicas: Un enfoque al sur del Perú, \textit{En "Construyendo Resiliencia Climática Agraria frente al Cambio Global en el Departamento del Cusco"}, Cusco, Perú.}
 
\cvitem{Nov. 2017}{\textbf{Huerta, A.}, (2017). Development of a daily gridded temperature product in Peru. \textit{En 7th HYdro-geochemistry of the AMazonian Basin (HYBAM) Scientific Meeting}, Niterói, Brasil. \href{https://hybam.obs-mip.fr/wp-content/uploads/2018/07/5_huerta.pdf}{https://hybam.obs-mip.fr/wp-content/uploads/2018/07/5\_huerta.pdf}.}

\cvitem{Abr. 2018}{\textbf{Huerta, A.}, (2018). Development of daily gridded precipitation and temperature in Peru. \textit{En South America Water from Space Conference 2018}, Santiago, Chile}

\cvitem{Jun. 2018}{\textbf{Huerta, A.}, (2018). Development of daily gridded temperature product in Peru: PISCOt v1.1, \textit{En Workshop Gestión de datos para servicios climáticos}, Lima, Perú.}

\cvitem{Nov. 2019}{\textbf{Huerta, A.}, (2019). PISCOt: a daily and monthly gridded air temperature dataset for Peru, \textit{En I Congreso Peruano de Meteorología}, Lima, Perú.}

\cvitem{Jul. 2020}{\textbf{Huerta, A.}, (2020). Climatología y tendencias de la temperatura del aire en el (sur) Perú, \textit{En I Simposio de Vivienda Rural y Productiva - Colegio de Ingenieros del Perú (CIP)}, Lima, Perú.}

\cvitem{Ag. 2022}{\textbf{Huerta, A.}, (2022). PISCOeo\_pm, una base de datos de evapotranspiración de referencia basada en FAO Penman-Monteith en Perú, \textit{En RAHU 2022: glaciares, seguridad hídrica y adaptación al cambio climático}, Cusco, Perú. \href{https://sites.google.com/utec.edu.pe/rahu-project/resultados}{https://sites.google.com/utec.edu.pe/rahu-project/resultados}}

\section{Entrenamiento extra académico}

\renewcommand{\listitemsymbol}{-~} % Changes the symbol used for lists

\cvitem{Nov. 2013}{Interpretación de imágenes satelitales. Instituto Nacional de Investigación Espacial de Brasil (INPE). 15 horas.}

\cvitem{Nov. 2013}{Modelización regional del Clima y aplicaciones en el Perú. Servicio Nacional de Meteorología e Hidrología del Perú (SENAMHI). 32 horas.}

\cvitem{Oct. 2013}{Variabilidad hidro-climática y modelos estocásticos en Hidrología. Universidad Nacional Agraria La Molina (UNALM). 30 horas.}

\cvitem{En. 2014}{SIG en la gestión del agua. Gidahatari. 20 horas}

\cvitem{Jun. 2014}{Meteorología costera. Servicio Nacional de Meteorología e Hidrología del Perú (SENAMHI). 40 horas.}

\cvitem{Oct. 2014}{Sensoramiento remoto aplicado a la hidrología de ríos amazónicos. Servicio Nacional de Meteorología e Hidrología del Perú (SENAMHI). 32 horas.}

\cvitem{Nov. 2014}{Programación en lenguaje R. PROMiDAT Iberoamericano. 20 horas.}

\cvitem{Nov. 2014}{Python en hidrología. Gidahatari. 20 horas}

\cvitem{Dic. 2014}{Geoestadística aplicado al mapeo de variables hidro-climáticas. Servicio Nacional de Meteorología e Hidrología del Perú (SENAMHI). 32 horas.}

\cvitem{Jun 2015}{Control de calidad, homogenización y metadata. Servicio Nacional de Meteorología e Hidrología del Perú (SENAMHI). 32 horas.}

\cvitem{Abr. 2015}{Introducción a SWAT. Servicio Nacional de Meteorología e Hidrología del Perú (SENAMHI). 24 horas.}

\cvitem{Jul. 2015}{Meteorología de montaña. Servicio Nacional de Meteorología e Hidrología del Perú (SENAMHI). 40 horas.}

\cvitem{Feb. 2016}{Análisis del producto satelital de precipitación GPM para la región andina. Servicio Nacional de Meteorología e Hidrología del Perú (SENAMHI) - Imperial College London. 10 horas.}

\cvitem{Jun. 2016}{Conducta responsable de investigación. Centro Andino de Investigación y Entrenamiento en Informática para la Salud Global (QUIPU). 10 horas.}

\cvitem{En. 2017}{The R Programming Environment. Universidad Johns Hopkins - Coursera. 20 horas.}

\cvitem{En. 2017}{Advanced R Programming. Universidad Johns Hopkins - Coursera. 20 horas.}

\cvitem{Feb. 2017}{Building R Packages. Universidad Johns Hopkins - Coursera. 20 horas.}

\cvitem{Mar. 2017}{Building Data Visualization Tools with R. Universidad Johns Hopkins - Coursera. 20 horas.}

\cvitem{Mar. 2017}{Bayesian Statistics: From Concept to Data Analysis. Universidad de California, Santa Cruz (UCSC) - Coursera. 20 horas.}

\cvitem{Oct. 2017}{Rescate de Datos Climáticos. Servicio Nacional de Meteorología e Hidrología del Perú (SENAMHI). 40 horas.}

\cvitem{Ag. 2018}{Modelo Hidrológico: Modelo de Grandes Cuencas MGB. Universidad Federal de Río Grande del Sur (UFRGS). 24 horas.}

\cvitem{Oct. 2018}{Procesamiento y análisis de datos geofísicos usando Python 3. Servicio Nacional de Meteorología e Hidrología del Perú (SENAMHI). 50 horas.}

\cvitem{May. 2019}{Python for Data Science and Machine Learning Bootcamp. UDEMY. 50 horas.}

\cvitem{Oct. 2020}{Capacitación en Estimación y Monitoreo de Precipitación por Radar. Instituto Geofísico del Perú (IGP) y Pontificia Universidad Católica del Perú (PUCP). 24 horas.}

\cvitem{En. 2021}{Epidemiología básica para instituciones de salud. Universidad de Chile. 27 horas.}

\cvitem{En. 2021}{Exposiciones ambientales y cáncer: evidencia epidemiológica. Universidad de Chile. 27 horas.}

\cvitem{Set. 2022 --}{Machine Learning Specialization. Universidad Stanford, DeepLearning.AI - Coursera.}