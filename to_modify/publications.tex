
\section{Asesoría de Tesis}

\cvitem{2020}{\textbf{M. Bojorquez}.
\href{http://repositorio.lamolina.edu.pe/handle/UNALM/2}{Evaluación del modelo ETA/SENAMHI durante eventos de nevadas intensas en la sierra sur del Perú}. Mención: Sobresaliente. Co-Supervisor: \textbf{A. Huerta}. Título: \textbf{Ingeniero Meteorólogo}.}

\cvitem{2019}{\textbf{S. Rivadeneira}. 
\href{http://repositorio.lamolina.edu.pe/handle/UNALM/4075}{Corrección de estimaciones de precipitación por satelite GPM-IMERG usando técnicas de mezcla sobre las cuencas Chillon-Rímac-Lurin}. Mención: Sobresaliente. Co-Supervisor: \textbf{A. Huerta}. Título: \textbf{Ingeniero Meteorólogo}.} 

\section{Publicaciones}

\renewcommand{\listitemsymbol}{-~} % Changes the symbol used for lists

\cvitem{2022}{\textbf{Huerta A.}, Aybar C., Imfeld N., Correa K., Felipe-Obando O., Lavado-Casimiro W. (2022). Update of the PISCOt gridded air temperature dataset over Peru. In preparation.}

\cvitem{}{\textbf{Huerta A.}, Lavado-Casimiro W., Felipe-Obando O. (2022). High-resolution gridded hourly precipitation dataset for Peru (PISCOp\_h). Submitted to Data in Brief Journal.}

\cvitem{}{Dávila J.E., Tapia V., Vasquez B.V., Anchiraico-Agudo W.R., \textbf{Huerta A.}, Chauca J., Gonzales G.F. (2022). Seasonality and meteorological factors in Acute Upper Respiratory Infections (AURIs) in children under 5 years old in Piura, Peru. Submitted to Journal of Environmental and Public Health.}

\cvitem{}{\textbf{Huerta A.}, Bonnesoeur, V., Cuadros, J., Gutierrez Lope, L. F., Ochoa-Tocachi, B., Román-Dañobeytia, F., Lavado-Casimiro, W. (2022). PISCOeo\_pm, a reference evapotranspiration gridded database based on FAO Penman-Monteith in Peru. Nature Scientific Data. \url{https://doi.org/10.1038/s41597-022-01373-8}}

\cvitem{}{Bojorquez M., \textbf{Huerta A.}, Calle V. (2022). A Case Study of a High Impact Snowfall Event in the Southern Andes of Peru: Dynamics and Evaluation of the Eta Model. Revista Brasileira de Meteorologia. \url{https://doi.org/10.1590/0102-7786360012}}

\cvitem{2021}{Delahoy M. J., Cárcamo C., \textbf{Huerta A.}, Lavado W., Escajadillo Y., Ordoñez L., Vasquez V., Lopman B., Clasen T., Gonzales G, Steenland K., and Levy K. (2021). Meteorological factors and childhood diarrhea in Peru, 2005–2015: a time series analysis of historic associations, with implications for climate change. Environ Health 20, 22. \url{https://doi.org/10.1186/s12940-021-00703-4}}

\cvitem{2020}{Imfeld N., Sedlmeir K., Gubler S., Correa K., Davila P., \textbf{Huerta A.}, Lavado W., Rohrer M., Scherrer S. and Schwierz C. (2020). A combined view on precipitation and temperature climatology and trends in the southern Andes of Peru. International Journal of Climatology. Accepted Author Manuscript. \url{https://doi.org/10.1002/joc.6645}}

\cvitem{}{\textbf{Huerta A.} and Lavado-Casimiro W. (2020). Trends and variability of precipitation extremes in the Peruvian Altiplano (1971-2013). International Journal of Climatology. Accepted Author Manuscript. \url{https://doi.org/10.1002/joc.6635}}

\cvitem{}{\textbf{Huerta A.}, Lavado, W., and Rau, P. (2020). The vulnerability of water availability in Peru due to climate change: A probabilistic Budyko analysis. \url{https://doi.org/10.5194/egusphere-egu2020-3766}}

\cvitem{}{Zevallos-Ruiz, J. A., \textbf{Huerta A.}, Lavado, W., Sabino, E., vega, F., and Felipe, O. (2020). Climate change impacts on biomes and aridity in Peru. \url{https://doi.org/10.5194/egusphere-egu2020-20432}}

\cvitem{}{Lavado-Casimiro, W., Jimenez, J. C., Llauca, H., Leon, K., Oria, C., Llacza, A., \textbf{Huerta A.}., Felipe, O., Acuña, J., Rau, P., and Abad, J. (2020). ANDES: The first system for flash flood monitoring and forecasting in Peru. \url{https://doi.org/10.5194/egusphere-egu2020-3759}}

\cvitem{}{Spirig, C., Gubler, S., Avalos, G., \textbf{Huerta A.}, Imfeld, N., Lavado, W., Oria, C., Quevedo, K., Rohrer, M., Scherrer, S. C., Sedlmeier, K., and Schwierz, C. (2020). Spatio-temporal temperature and precipitation patterns in the southern Peruvian Andes—Insights from the Climandes project. \url{https://doi.org/10.5194/egusphere-egu2020-14175}}

\cvitem{2019}{Aybar, C., Fernández, C., \textbf{Huerta A.}, Lavado, W., Vega, F., and Felipe-Obando, O. (2019). Construction of a high-resolution gridded rainfall dataset for Peru from 1981 to the present day. Hydrological Sciences Journal, 65(5), 770-785. \url{https://doi.org/10.1080/026266 7.2019.1649411}}

\cvitem{}{Imfeld, N., Barreto Schuler C., Correa Marrou K. M., Jacques-Coper M., Sedlmeier K., Gubler S., \textbf{Huerta, A.}, and Brönnimann S. (2019). Summertime precipitation deficits in the southern Peruvian highlands since 1964. International Journal of Climatology, 39, 4497–4513. \url{https://doi.org/10.1002/joc.6087}}

\cvitem{2018}{\textbf{Huerta A.}, Aybar C., Lavado W. (2018). PISCO temperatura v1.1. SENAMHI-DHI-2018.}

\cvitem{}{Andrade, M. F., Moreno I., Calle J. M. , Ticona L.,  Blacutt L., Lavado-Casimiro W., Sabino E., \textbf{Huerta A.}, Aybar C., Hunziker S., and Brönnimann S. (2018). Atlas - Clima y eventos extremos del Altiplano Central perú-boliviano / Climate and extreme events from the Central Altiplano of Peru and Bolivia / 1981-2010. Geographica Bernensia, 188 pp. \url{https://doi.org/10.4480/GB2018.N01}}

\cvitem{}{Imfeld, N., \textbf{Huerta A.}, and Lavado W. (2018) La sequía de 1982-83 en el Altiplano / The 1982-83 drought in the Altiplano, in: Andrade, M. F., et al. (Eds). Atlas - Clima y eventos extremos del Altiplano Central perú-boliviano / Climate and extreme events from the Central Altiplano of Peru and Bolivia / 1981-2010. Geographica Bernensia, p. 74-75. \url{https://doi.org/10.4480/GB2018.N01}}

\cvitem{}{Hunziker, S., Brönnimann, S., Calle, J., Moreno, I., Andrade, M., Ticona, L., \textbf{Huerta, A.}, and Lavado-Casimiro, W. (2018) Effects of undetected data qual-ity issues on climatological analyses. Climate of the Past, 14(1), 1–20. \url{https://doi.org/10.5194/cp-14-1-2018}}

\section{Datos}

\cvitem{2021}{\textbf{Huerta A.}, Lavado-Casimiro, W., Felipe-Obando O.. (2021). Development of high-resolution hourly gridded precipitation dataset over Peru. \url{https://figshare.com/collections/Development\_of\_high-resolution\_hourly\_gridded\_precipitation\_dataset\_over\_Peru/5743166}}

\cvitem{}{\textbf{Huerta A.}, Bonnesoeur, V., Cuadros, J., Gutierrez Lope, L. F., Ochoa-Tocachi, B., Román-Dañobeytia, F., and Lavado-Casimiro, W. (2021). A reference evapotranspiration gridded database based on FAO Penman-Monteith in Peru during 1981-2016. \url{https://figshare.com/projects/A\_gridded\_reference\_evapotranspiration\_dataset\_based\_on\_FAO\_Penman-Monteith\_in\_Peru\_during\_1981-2016/120738}}

\cvitem{2020}{\textbf{Huerta A.} A gridded annual runoff dataset from 1982 to 2016 for Peru figshare. Dataset. \url{https://doi.org/10.6084/m9.figshare.13404413}}

\cvitem{}{\textbf{Huerta A.} Hourly gridded precipitation product in (CHI)llon, (RI)mac and (LU)rin basins (CHIRILU v2) (2020). figshare. Dataset. \url{https://doi.org/10.6084/m9.figshare.13260020}}

\cvitem{}{\textbf{Huerta A.} (2020): Actual evapotranspiration at Peru scale (2003-2013). figshare. Dataset. \url{https://doi.org/10.6084/m9.figshare.13270391}}

\cvitem{2019}{\textbf{Huerta A.} PISCOpet v1.0: daily/monthly potential evapotranspiration from 1981-2016 at Peru scale (2019). \url{http://iridl.ldeo.columbia.edu/SOURCES/.SENAMHI/.HSR/.PISCO/.PET/}}

\cvitem{}{Aybar, C., Fernández, C., \textbf{Huerta A.}, Lavado, W., Vega, F., and Felipe-Obando, O. (2019). PISCOp v2.0: daily/monthly gridded precipitation from 1981-2016 at Peru scale. \url{http://iridl.ldeo.columbia.edu/SOURCES/.SENAMHI/.HSR/.PISCO/.Prec/}}

\cvitem{2018}{\textbf{Huerta A.}, Aybar C., Lavado W. (2018). PISCOt v1.0: daily/monthly gridded air temperature from 1981-2016 at Peru scale. \url{http://iridl.ldeo.columbia.edu/SOURCES/.SENAMHI/.HSR/.PISCO/.Temp/}}
