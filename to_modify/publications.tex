\section{Publicaciones}
\subsection{\textit{En revistas científicas internacionales revisadas por pares}}
\renewcommand{\listitemsymbol}{-} % Changes the symbol used for lists

\cvitem{2018}{Hunziker, S., Brönnimann, S., Calle, J., Moreno, I., Andrade, M., Ticona, L., \textbf{Huerta, A.}, Lavado-Casimiro, W. (2018). Effects of undetected data quality issues on climatological analyses. Climate of the Past, 14(1), 1-20.
\href{https://doi.org/10.5194/cp-14-1-2018}{https://doi.org/10.5194/cp-14-1-2018}}

\cvitem{2019}{Imfeld, N., Barreto Schuler C., Correa Marrou K. M., Jacques-Coper M., Sedlmeier K., Gubler S., \textbf{Huerta, A.}, and Brönnimann S. (2019). Summertime precipitation deficits in the southern Peruvian highlands since 1964. International Journal of Climatology, 39(11), 4497–4513. 
\href{https://doi.org/10.1002/joc.6087}{https://doi.org/10.1002/joc.6087}}

\cvitem{}{Aybar, C., Fernández, C., \textbf{Huerta, A.}, Lavado, W., Vega, F., and Felipe-Obando, O. (2019). Construction of a high-resolution gridded rainfall dataset for Peru from 1981 to the present day. Hydrological Sciences Journal, 65(5), 770-785.
\href{https://doi.org/10.1080/02626667.2019.1649411}{https://doi.org/10.1080/02626667.2019.1649411}}

\cvitem{2021}{\textbf{Huerta, A.}, and Lavado‐Casimiro, W. (2021). Trends and variability of precipitation extremes in the Peruvian Altiplano (1971–2013). International Journal of Climatology, 41(1), 513-528.
\href{https://doi.org/10.1002/joc.6635}{https://doi.org/10.1002/joc.6635}}

\cvitem{}{Imfeld, N., Sedlmeier, K., Gubler, S., Correa Marrou, K., Davila, C. P., \textbf{Huerta, A.}, Lavado-Casimiro, W., Rohrer, M., Scherrer, S.C., and Schwierz, C. (2021). A combined view on precipitation and temperature climatology and trends in the southern Andes of Peru. International journal of climatology, 41(1), 679-698.
\href{https://doi.org/10.1002/joc.6645}{https://doi.org/10.1002/joc.6645}}

\cvitem{}{Delahoy M. J., Cárcamo C., \textbf{Huerta, A.}, Lavado W., Escajadillo Y., Ordoñez L., Vasquez V., Lopman B., Clasen T., Gonzales G, Steenland K., and Levy K. (2021). Meteorological factors and childhood diarrhea in Peru, 2005–2015: a time series analysis of historic associations, with implications for climate change. Environmental Health 20(1), 1-10. \href{https://doi.org/10.1186/s12940-021-00703-4}{https://doi.org/10.1186/s12940-021-00703-4}}

\cvitem{2022}{Bojorquez, M., \textbf{Huerta, A.} and Calle, V. (2022). A Case Study of a High Impact Snowfall Event in the Southern Andes of Peru: Dynamics and Evaluation of the Eta Model. Revista Brasileira de Meteorologia, 37(1). \href{https://doi.org/10.1590/0102-7786360012}{https://doi.org/10.1590/0102-7786360012}}

\cvitem{}{\textbf{Huerta, A.}, Bonnesoeur, V., Cuadros, J., Gutierrez Lope, L. F., Ochoa-Tocachi, B., Román-Dañobeytia, F., Lavado-Casimiro, W. (2022). PISCOeo\_pm, a reference evapotranspiration gridded database based on FAO Penman-Monteith in Peru. Nature Scientific Data, 9(1), 1-18.
\href{https://doi.org/10.1038/s41597-022-01373-8}{https://doi.org/10.1038/s41597-022-01373-8}}

\cvitem{}{\textbf{Huerta, A.}, Lavado-Casimiro, W., and Felipe-Obando, O. (2022). High-resolution gridded hourly precipitation dataset for Peru (PISCOp\_h). Data in Brief, 108570. \href{https://doi.org/10.1016/j.dib.2022.108570}{https://doi.org/10.1016/j.dib.2022.108570}}

\cvitem{}{Dávila, J.E., Tapia, V., Vasquez, B.V., Anchiraico-Agudo, W.R., \textbf{Huerta, A.}, Chauca, J. and Gonzales, G.F. Seasonality and meteorological factors in Acute Upper Respiratory Infections (AURIs) in children under 5 years old in Piura, Peru. In preparation for the Journal of Environmental and Public Health.}

\cvitem{}{\textbf{Huerta, A.}, Aybar, C., Imfeld N., Correa K., Felipe-Obando O., Rau, P., Drenkhan, F. and Lavado-Casimiro W. High-resolution grids of daily air temperature for Peru - the PISCOt v1.2 dataset. In preparation for Nature Scientific Data.}

\cvitem{}{Gutierrez, L., Lavado-Casimiro, W., Sabino, E. and \textbf{Huerta, A}. Satellite-based estimation of rainfall erosivity for Peru: Spatio temporal evaluation 2000-2020. In preparation for Remote Sensing.}

\subsection{\textit{Libros/monografías}}

\cvitem{2017}{Aybar, C., Lavado-Casimiro, W., Sabino, E., Ramírez S., \textbf{Huerta, A.}, y Felipe-Obando, O., (2017). Atlas
de zonas de vida del Perú – Guía Explicativa. Servicio Nacional de Meteorología e Hidrología del Perú. \href{https://repositorio.senamhi.gob.pe/handle/20.500.12542/259}{https://repositorio.senamhi.gob.pe/handle/20.500.12542/259}}

\cvitem{}{Aybar, C., Lavado-Casimiro, W., \textbf{Huerta, A.}, Fernández, C., Vega, F., Sabino, E., y Felipe-Obando, O., (2017). Uso del Producto Grillado PISCO de precipitación en Estudios, Investigaciones y Sistemas Operacionales de Monitoreo y Pronóstico Hidrometeorológico.  Servicio Nacional de Meteorología e Hidrología del Perú. \href{https://repositorio.senamhi.gob.pe/handle/20.500.12542/260}{https://repositorio.senamhi.gob.pe/handle/20.500.12542/260}}

\cvitem{2018}{Andrade, M. F., Moreno, I., Calle, J. M. , Ticona, L.,  Blacutt, L., Lavado-Casimiro, W., Sabino, E., \textbf{Huerta, A.}, Aybar, C., Hunziker, S., and Brönnimann, S. (2018). Atlas - Clima y eventos extremos del Altiplano Central perú-boliviano / Climate and extreme events from the Central Altiplano of Peru and Bolivia / 1981-2010. Geographica Bernensia, 188 pp. \href{https://doi.org/10.4480/GB2018.N01}{https://doi.org/10.4480/GB2018.N01}}

\cvitem{}{\textbf{Huerta, A.}, Aybar, C., Lavado-Casimiro W. (2018). PISCO temperatura v1.1. Servicio Nacional de Meteorología e Hidrología del Perú. \href{http://iridl.ldeo.columbia.edu/documentation/.pisco/.PISCOt_report.pdf}{http://iridl.ldeo.columbia.edu/documentation/.pisco/.PISCOt\_report.pdf}}

\cvitem{2020}{Lavado-Casimiro, W., Llauca, H., Montesinos, C., Asencios, H., Ordoñez, J., Sosa, J., Yali, R., Tupac-Yupanqui, R., Quijada, N., Asurza, F., Traverso, K., \textbf{Huerta, A.}, Sabino, E. and Vega, F., Estudios Hidrológicos del SENAMHI: Resúmenes Ejecutivos - 2020. Servicio Nacional de Meteorología e Hidrología del Perú. \href{https://www.senamhi.gob.pe/load/file/01401SENA-90.pdf}{https://www.senamhi.gob.pe/load/file/01401SENA-90.pdf}.}

\cvitem{2021}{\textbf{Huerta, A.}, and Lavado-Casimiro, W. (2021). Atlas de zonas áridas del Perú: una evaluación presente y futura. Servicio Nacional de Meteorología e Hidrología del Perú. \href{https://hdl.handle.net/20.500.12542/1206}{https://hdl.handle.net/20.500.12542/1206}}

\cvitem{}{\textbf{Huerta, A.}, and Lavado-Casimiro, W. (2021). Atlas de producción de agua en el Perú: una evaluación presente y futura con énfasis en las cuencas de aporte de la EPS. Servicio Nacional de Meteorología e Hidrología del Perú.
\href{https://hdl.handle.net/20.500.12542/1610}{https://hdl.handle.net/20.500.12542/1610}}

\subsection{\textit{Conferencias revisadas por pares}}

\cvitem{2020}{\textbf{Huerta, A.}, Lavado-Casimiro, W., and Rau, P. (2020). The vulnerability of water availability in Peru due to climate change: A probabilistic Budyko analysis. \href{https://doi.org/10.5194/egusphere-egu2020-3766}{https://doi.org/10.5194/egusphere-egu2020-3766}}

\cvitem{}{Zevallos-Ruiz, J. A., \textbf{Huerta, A.}, Lavado-Casimiro, W., Sabino, E., vega, F., and Felipe, O. (2020). Climate change impacts on biomes and aridity in Peru. \href{https://doi.org/10.5194/egusphere-egu2020-20432}{https://doi.org/10.5194/egusphere-egu2020-20432}}

\cvitem{}{Lavado-Casimiro, W., Jimenez, J. C., Llauca, H., Leon, K., Oria, C., Llacza, A., \textbf{Huerta, A.}., Felipe, O., Acuña, J., Rau, P., and Abad, J. (2020). ANDES: The first system for flash flood monitoring and forecasting in Peru. \href{https://doi.org/10.5194/egusphere-egu2020-3759}{https://doi.org/10.5194/egusphere-egu2020-20432}}

\cvitem{}{Spirig, C., Gubler, S., Avalos, G., \textbf{Huerta, A.}, Imfeld, N., Lavado, W., Oria, C., Quevedo, K., Rohrer, M., Scherrer, S. C., Sedlmeier, K., and Schwierz, C. (2020). Spatio-temporal temperature and precipitation patterns in the southern Peruvian Andes—Insights from the CLIMANDES project. \href{https://doi.org/10.5194/egusphere-egu2020-14175}{https://doi.org/10.5194/egusphere-egu2020-14175}}

\subsection{\textit{Contribuciones a libros}}

\cvitem{2018}{Imfeld, N., \textbf{Huerta, A.}, and Lavado-Casimiro, W. (2018) La sequía de 1982-83 en el Altiplano / The 1982-83 drought in the Altiplano, in: Andrade, M. F., et al. (Eds). Atlas - Clima y eventos extremos del Altiplano Central perú-boliviano / Climate and extreme events from the Central Altiplano of Peru and Bolivia / 1981-2010. Geographica Bernensia, p. 74-75. \href{https://doi.org/10.4480/GB2018.N01}{https://doi.org/10.4480/GB2018.N01}}

\cvitem{2019}{Rau, P.,  Buytaert, W., Drenkhan, F., Lavado-Casimiro, W., Montoya, N., Gianella, C., Goyburo, A., Risco, E., Cachay, W., Abad, J., Jiménez, J.C., Suarez, W., \textbf{Huerta, A.}, Baca, C., Macera, B., Bueno, M., Bonnesoeur, V. and Valdivia., G., (2019). RAHU: Implications of glacier shrinkage on future tropical Andean water security and management. I Simposio Glaciares Tropicales. WEATHER: a scientific approach in Water sEcurity and climATe cHange adaptation in pEruvian glacieRs - Libro de resúmenes. \href{https://app.ingemmet.gob.pe/biblioteca/pdf/TGS-I.pdf}{https://app.ingemmet.gob.pe/biblioteca/pdf/TGS-I.pdf}}

\cvitem{}{\textbf{Huerta, A.}, Lavado-Casimiro, W., Jiménez, J.C., (2019). Updated high-resolution grids of monthly air temperature observations - PISCOt v1.2.  - Full abstract book. Simposio Científico "Las Montañas, Nuestro Futuro" - Libro de resúmenes.}

\cvitem{2020}{Rojas, I., Suarez, W., Loarte, E., Yarleque, C., Vega, F., \textbf{Huerta, A.} and Davila, L. (2020). Glacier retreat and ocean-atmosphere interactions at King George Island - Antarctic Peninsula. Scientific Committee on Antarctic Research (SCAR) Open Science Conference 2020 - Full abstract book. \href{https://www.scar.org/science-meetings/osc/open-science-conference-abstracts/5534-scar-osc-2020-abstracts/file/}{ISBN: 978-0-948277-59-7}}

\subsection{\textit{Conjuntos de datos}}

\cvitem{2018}{\textbf{Huerta, A.}, Aybar C., Lavado-Casimiro W. (2018). PISCOt v1.1: temperatura del aire grillada a escala diaria/mensual de 1981 a 2016 a nivel de Perú. \href{http://iridl.ldeo.columbia.edu/SOURCES/.SENAMHI/.HSR/.PISCO/.Temp/}{http://iridl.ldeo.columbia.edu/SOURCES/.SENAMHI/.HSR/.PISCO/.Temp/}}

\cvitem{2019}{Aybar, C., Fernández, C., \textbf{Huerta, A.}, Lavado, W., Vega, F., and Felipe-Obando, O. (2019). PISCOp v2.1: precipitación grillada a escala diaria/mensual de 1981-2016 a nivel de Perú. \href{http://iridl.ldeo.columbia.edu/SOURCES/.SENAMHI/.HSR/.PISCO/.Prec/}{http://iridl.ldeo.columbia.edu/SOURCES/.SENAMHI/.HSR/.PISCO/.Prec/}}

\cvitem{}{\textbf{Huerta, A.} PISCOpet v1.0: evapotranspiración potencial a escala diaria/mensual de 1981-2016 a nivel de Perú (2019). \href{http://iridl.ldeo.columbia.edu/SOURCES/.SENAMHI/.HSR/.PISCO/.PET/}{http://iridl.ldeo.columbia.edu/SOURCES/.SENAMHI/.HSR/.PISCO/.PET/}}

\cvitem{2020}{\textbf{Huerta, A.} (2020). Evapotranspiración real a escala Perú (2003-2013).
\href{https://doi.org/10.6084/m9.figshare.13270391}{https://doi.org/10.6084/m9.figshare.13270391}}

\cvitem{}{\textbf{Huerta, A.} (2020). Un conjunto de datos grillados de escurrimiento anual desde 1982 a 2016 para Perú. \href{https://doi.org/10.6084/m9.figshare.13404413}{https://doi.org/10.6084/m9.figshare.13404413}} 

\cvitem{}{\textbf{Huerta, A.} (2020). Producto de precipitación grillada por hora en las cuencas (CHI)llon, (RI)mac y (LU)rin (CHIRILU v2). \href{https://doi.org/10.6084/m9.figshare.13260020}{https://doi.org/10.6084/m9.figshare.13260020}}

\cvitem{2021}{\textbf{Huerta, A.}, Bonnesoeur, V., Cuadros, J., Gutierrez Lope, L. F., Ochoa-Tocachi, B., Román-Dañobeytia, F., and Lavado-Casimiro, W. (2021). PISCOeo\_pm, a reference evapotranspiration gridded database based on FAO Penman-Monteith in Peru. \href{https://doi.org/10.6084/m9.figshare.c.5633182}{https://doi.org/10.6084/m9.figshare.c.5633182}}

\cvitem{}{\textbf{Huerta, A.}, Gutierrez Lope, L. F., Lavado-Casimiro, W., y Sabino Rojas, E. D. (2021). Atlas de zonas áridas del Perú: Una evaluación presente y futura. \href{https://doi.org/10.6084/m9.figshare.14067035}{https://doi.org/10.6084/m9.figshare.14067035}}

\cvitem{}{\textbf{Huerta, A.} (2021). Atlas de producción de agua en el Perú: una evaluación presente y futura con énfasis en las cuencas de aporte de las EPS.
\href{https://doi.org/10.6084/m9.figshare.17162087}{https://doi.org/10.6084/m9.figshare.17162087}}

\cvitem{2022}{\textbf{Huerta, A.}, Lavado-Casimiro, W., and Felipe-Obando, O. (2022). High-resolution gridded hourly precipitation dataset for Peru (PISCOp\_h).
\href{https://doi.org/10.6084/m9.figshare.c.5743166}{https://doi.org/10.6084/m9.figshare.c.5743166}}

\cvitem{}{\textbf{Huerta, A.}, Aybar, C., Imfeld N., Correa K., Felipe-Obando O., Rau, P., Drenkhan, F. and Lavado-Casimiro W. (2022). High-resolution grids of daily air temperature for Peru - the PISCOt v1.2 dataset. \href{https://doi.org/10.6084/m9.figshare.c.5959863}{https://doi.org/10.6084/m9.figshare.c.5959863}}

\section{Actividades científicas}

\cvitem{2021 --}{Revisor de artículos científicos (International Journal of Climatology, Big Earth Data)}

\section{Tesis supervisadas o co-supervisadas}
\subsection{\textit{Tesis de grado en ingeniería}}

\cvitem{Jun. 2019}{Rivadeneira S. "Corrección de estimaciones de precipitación por satélite GPM-IMERG usando técnicas de mezcla sobre las cuencas Chillon-Rímac-Lurin". Universidad Nacional Agraria La Molina. Obtuvo excelente calificación. \href{http://repositorio.lamolina.edu.pe/handle/20.500.12996/4075}{http://repositorio.lamolina.edu.pe/handle/20.500.12996/4075}}

\cvitem{Dic. 2020}{Bojorquez, M. "Evaluación del modelo ETA/SENAMHI durante eventos de nevadas intensas en la sierra sur del Perú". Universidad Nacional Agraria La Molina. Obtuvo excelente calificación. \href{http://repositorio.lamolina.edu.pe/handle/20.500.12996/4701}{http://repositorio.lamolina.edu.pe/handle/20.500.12996/4701}}