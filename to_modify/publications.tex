\section{Publicaciones}

\renewcommand{\listitemsymbol}{-~} % Changes the symbol used for lists

\cvitem{2019}{Aybar C., Fernandez C, \textbf{Huerta A.}, Lavado W., Vega F. and Felipe O. (2019). PISCOp: A gridded daily and monthly rainfall dataset for Peru from 1981 to the present. Submitted to Hydrological Processes.}

\cvitem{}{\textbf{Huerta A.} and Lavado W. (2019). Trends and variability of precipitation extremes in the Peruvian Altiplano during 1971-2013. In preparation.}

\cvitem{2018}{\textbf{Huerta A.}, Aybar C., Lavado W. (2018). PISCO temperatura v1.1. SENAMHI-DHI-2018.}

\cvitem{}{Imfeld N., Correa K., Barreto-Schuler C., Jacques-Coper M., Sedlmeir K., Gubler S., \textbf{Huerta A.} and Brönnimann S. (2018). Summertime precipitation deficits in the southern Peruvian highlands since 1964. Submitted to International Journal of Climatology.}

\cvitem{}{Andrade, M. F., Moreno I., Calle J. M. , Ticona L.,  Blacutt L., Lavado-Casimiro W., Sabino E., \textbf{Huerta A.}, Aybar C., Hunziker S., and Brönnimann S. (2018) Atlas - Clima y eventos extremos del Altiplano Central perú-boliviano / Climate and extreme events from the Central Altiplano of Peru and Bolivia / 1981-2010. Geographica Bernensia, 188 pp., doi: 10.4480/GB2018.N01.}

\cvitem{}{Imfeld, N., \textbf{Huerta A.}, and Lavado W. (2018) La sequía de 1982-83 en el Altiplano / The 1982-83 drought in the Altiplano, in: Andrade, M. F., et al. (Eds). Atlas - Clima y eventos extremos del Altiplano Central perú-boliviano / Climate and extreme events from the Central Altiplano of Peru and Bolivia / 1981-2010. Geographica Bernensia, p. 74-75., doi: 10.4480/GB2018.N01.}

\cvitem{}{Hunziker, S., Brönnimann, S., Calle, J., Moreno, I., Andrade, M., Ticona, L., \textbf{Huerta, A.}, and Lavado-Casimiro, W. (2018). Effects of undetected data quality issues on climatological analyses, Clim. Past, 14, 1-20, https://doi.org/10.5194/cp-14-1-2018.}

\cvitem{2016}{Lavado-Casimiro, W. S., Aybar, C., \textbf{Huerta, A.}, Sabino, E., Zevallos, J. y Felipe-Obando O. (2016). Generación de datos grillados de precipitación diaria (PISCO Pd 1981-2015) y su utilidad para la estimación de umbrales de precipitaciones máximas. Servicio Nacional de Meteorología e Hidrología del Perú (SENAMHI). 	Estudio de la Dirección de Hidrología.}