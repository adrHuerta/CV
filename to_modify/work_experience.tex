\section{Experiencia laboral}

\cventry{Jun. 2013 -- Dic. 2013}{Practicas}{Servicio Nacional de Meteorología e Hidrología del Perú (SENAMHI)}{Área de Climatología}{Perú}{}

\cventry{En. 2014 -- Jun. 2016}{Asistente de investigación}{Servicio Nacional de Meteorología e Hidrología del Perú (SENAMHI)}{Área de Hidrología}{Perú}{Dentro del proyecto \href{https://www.geography.unibe.ch/research/climatology_group/research_projects/decade/index_eng.html}{Data on climate and Extreme weather for the Central Andes (DECADE)}.}

\cventry{Ag. 2016 -- Dic. 2019}{Asistente de investigación}{Servicio Nacional de Meteorología e Hidrología del Perú (SENAMHI)}{Área de Hidrología}{Perú}{Dentro del proyecto \href{https://www.meteoswiss.admin.ch/home/research-and-cooperation/international-cooperation/international-projects/climandes.subpage.html/en/data/projects/2016/climandes-2.html}{Servicios CLIMáticos con énfasis en los ANdes en apoyo a las DEcisioneS (CLIMANDES)} fase 2.}

\cventry{Oct. 2019 -- Dic. 2019}{Consultor de investigación}{HELVETAS Intercooperación suiza}{}{Perú}{Dentro del proyecto \href{https://www.helvetas.org/es/peru/lo-que-hacemos/como-trabajamos/nuestros-proyectos/America-latina/Peru/peru-euroclimaplus}{Pachayachay/Pachayatiña}.}

\cventry{May. 2019 -- Jul. 2021}{Consultor de investigación}{Instituto Libélula para el Cambio Global}{}{Perú}{Dentro del proyecto \href{https://www.proyectoapoyocambioclimatico.pe/}{Apoyo a la Gestión del Cambio Climático (GestionCC)} fase 2.}

\cventry{May. 2019 -- Ag. 2022}{Consultor de investigación}{Universidad de Ingeniería y Tecnología (UTEC)}{}{Perú}{Dentro del proyecto \href{https://gtr.ukri.org/projects?ref=NE\%2FS013210\%2F1}{WateR security And climate cHange adaptation in PerUvian glacier-fed river basins (RAHU)}.}

\cventry{Ag. 2020 -- Dic. 2020}{Consultor de investigación}{Superintendencia Nacional de Servicios de Saneamiento (SUNASS)}{}{Perú}{Producción del rendimiento de agua en Perú utilizando el marco Budyko.}

\cventry{Ag. 2020 -- Nov. 2020}{Consultor de investigación}{Servicio Nacional de Meteorología e Hidrología del Perú (SENAMHI)}{Área de Hidrología}{Perú}{Desarrollo de datos grillados de precipitación por hora en las cuencas de Chillón, Rimac y Lurín.}

\cventry{Dic. 2020 -- Jul. 2021}{Consultor de investigación}{Forest Trends}{}{Perú}{Dentro del proyecto \href{https://www.forest-trends.org/who-we-are/initiatives/water-initiative/natural-infrastructure-for-water-security-in-peru/}{Natural Infrastructure for Water Security (NIWS)}.}

\cventry{Oct. 2021 -- Dic. 2021}{Consultor de investigación}{Superintendencia Nacional de Servicios de Saneamiento (SUNASS)}{}{Perú}{Impacto del cambio climático en la producción de agua en Perú utilizando el marco Budyko.}

\cventry{Jun. 2021 -- Dic. 2021}{Consultor de investigación}{Servicio Nacional de Meteorología e Hidrología del Perú (SENAMHI)}{Área de Hidrología}{Perú}{Desarrollo de datos grillados de precipitación horaria a escala de Perú.}

\cventry{En. 2022 -- Jun. 2022}{Consultor de investigación}{Compañía Minera Antapaccay - GLENCORE}{}{Perú}{Caracterización climática y evaluación del impacto del cambio climático asociado a la seguridad de las presas.}

\cventry{Mar. 2022 -- Jun. 2022}{Consultor de investigación}{Autoridad Nacional del Agua (ANA)}{}{Perú}{Análisis del cambio climático en la disponibilidad de agua en seis cuencas hidrográficas.}

\cventry{Jun. 2022 -- }{Consultor de investigación}{Servicio Nacional de Meteorología e Hidrología del Perú (SENAMHI)}{Área de Hidrología}{Perú}{Dentro del proyecto \href{https://public.wmo.int/en/projects/enhancing-adaptive-capacity-of-andean-communities-through-climate-services-enandes}{Enhancing Adaptive Capacity of Andean Communities through Climate Services (ENANDES)}.}


\section{Experiencia como docente}

\cvitem{Nov. 2019}{"R aplicado a la hidrología" - ANDES Engineers \& projects -- 16 horas}
\cvitem{Ag. 2020}{"R aplicado a la hidro-meteorología" - Servicio Nacional de Meteorología e Hidrología del Perú (SENAMHI) -- 28 horas}
