\section{Experiencia}

%\subsection{Vocational}

\cventry{2019-}{Asistente de investigación}{\textsc{Libelula - SENAMHI}}{Lima}{}{En el marco del Proyecto GestionCC 2da. Fase - \textit{Apoyo a la Gestión del Cambio Climático}. Evaluación del impacto del cambio climático en la variabilidad espacio-temporal del indice de aridez en el Perú.}

\cventry{2019-}{Asistente de investigación}{\textsc{CITA-UTEC}}{Lima}{}{En el marco del Proyecto RAHU - \textit{Water security and climate change adaptation in glacier-fed river basins in Peru.} Apoyo en el diseño e implementación del monitoreo hidro-glaciológico para cuantificar las contribucciones no glaciares a las fuentes del agua e impacto de intervenciones antropogenicas. Cuencas del rio Vilcanota-Urubamba.}

\cventry{2019}{Asistente de investigación}{\textsc{Helvetas-SENAMHI}}{Lima}{}{En el marco del Proyecto Pachayachay/Pachayatiña - \textit{Información, gobernanza y acción para la reducción del riesgo de sequías en Perú y Bolivia en un contexto de cambio climático.} Apoyo en el desarrollo de métodos de completación y grillado de datos; estimación de evapotranspiración y monitoreo de sequías.}

\cventry{2016--2019}{Asistente de investigación}{\textsc{SENAMHI - Dirección de Hidrología}}{Lima}{}{En el marco del Proyecto CLIMANDES 2 - \textit{Servicios Climáticos con énfasis en los Andes en apoyo a las Decisiones}. Análisis y evaluación de diferentes métodos de combinación entre datos satelitales y observados (precipitación y temperatura) para la construcción de bases de datos grillados a escala diaria.}

\cventry{2014--2016}{Asistente de investigación}{\textsc{SENAMHI - Dirección de Hidrología}}{Lima}{}{En el marco del Proyecto DECADE - \textit{Datos climáticos y eventos extremos para el área central de los Andes}. Análisis y evaluación de diferentes métodos de control de calidad y de homogenización de datos climáticos, cálculo de índices de extremos y su relación con diferentes variables atmosféricas}

%------------------------------------------------
\cventry{2013}{Practicante}{\textsc{SENAMHI - Dirección de Climatología}}{Lima}{}{Practicas pre-profesionales. Tema de investigación: Nichos Climáticos.}

%------------------------------------------------

%\subsection{Miscellaneous}


